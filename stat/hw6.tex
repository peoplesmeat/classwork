\documentclass[11pt,fleqn]{article}
\usepackage{latexsym}
\usepackage{amsmath,amsthm}

\begin{document}
\newcommand{\mbf}[1]{\mbox{{\bfseries #1}}}
\newcommand{\N}{\mbf{N}}
\renewcommand{\O}{\mbf{O}}
\newcommand{\tabfrac}[2]{%
	\setlength{\fboxrule}{0pt}%
	\fbox{$\frac{#1}{#2}$}%
}

\noindent Bill Davis 

\noindent October 15th 2009 

\noindent Homework 6

\begin{enumerate}
  
  \item[5.2.1]
  For binomials the parameter $k$ can be determined by 
  \[
  k = \left(\frac{1}{n}\right)\sum_{u=1}^n x_i = \frac{5}{8}
  \]
  \item[5.2.3]
  Here $f_Y(y;\lambda) = \lambda e^{-\lambda y}$ 
  \[
  L(\lambda) = \displaystyle\prod_{i=1}^n \lambda e^{-\lambda y_i} = \lambda^n
  e^{-\left(\lambda \sum_{i=1}^n y_i)\right)}
  \]
  \[
  $ln$\, L(\lambda) = n $ln$\, \lambda - \lambda \displaystyle\sum_{i=1}^n y_i
  \]
  Taking the derivative
  \[
  \frac{d ln\, L(\lambda)}{d\lambda} = \frac{n}{\lambda} -
  \displaystyle\sum_{i=1}^n y_i
  \]
  And setting this to 0 
  \[
  \lambda = \frac{n}{\displaystyle\sum_{i=1}^{n} y_i} =
  \frac{4}{8.2+9.1+10.6+4.9}  = 0.098039
  \]
  \item[5.2.21]
  In this case there is only a single parameter to estimate so we only need one
  equation. The sample moment,  $E(y_1)$ is $\frac{1}{5}(1+1) = \frac{2}{5}$.
  \[
  E(y_1) = \theta^1(1-\theta)^0 = \frac{2}{5} = \theta
  \]
  Then $\theta = \frac{2}{5}$
  \item[5.3.1]
  
  The 95\% confidence interval for this problem is 
  \[
  0.8 - \frac{1.96\cdot0.09}{\sqrt{19}}, 0.8 + \frac{1.96\cdot0.09}{\sqrt{19}} 
  \]
  \[
  (0.759531,0.840869)
  \]
  Where the mean of the samples is $\frac{14.56}{19} = 0.766315$. Since this is
  inside the range the hypothesis that the bacteria has no effect is confirmed. 
  \item[5.3.4]
  \begin{enumerate}
      \item 
      	0.9900 - 0.0505 = 0.939594
      \item
      	0.99506 - 0 = 0.99506
      \item 
      	0.5 - 0.0505026 = 0.449497
  \end{enumerate}
  \item[5.3.5]
  The confidence interval for this experiment is 0.8554 - 0.1685 = 0.6869. The
  probabality that at least 4 of the intervals will contain the unknown mean is 
  \[
  0.6869^5 + (0.6869^4)(0.3131) = 0.222625
  \]
  \item[5.3.6]
  This interval is the only 95\% interval that is symmetric, that is the real
  value has an equal probabality of being below the estimator as above it.
  
  \item[5.4.1]
  In order for $| \hat{\theta} -3 | > 1 $ , the sum of the 2 chips must be
  greater then 8 or less then 4. The chips which deliver this total are
  (5,5),(5,4),(4,5),(1,1),(1,2),(2,1). Then the probabality of drawing these
  cards is $\frac{6}{25}$
  \item[5.4.2]
  \begin{enumerate}
      \item
      	In this case the probabality that the $\hat{\theta}$ is within 0.2 of 3
      	is the case where at least one sample is greater then or equal to 2.8.
      	That is 1 - P(All Samples $<$ 2.8), Where the probabality that a sample
      	is less then 2.8 is
      	\[
      		\int_0^{2.8} \frac{1}{3}\,dy = \frac{y}{3}|_0^{2.8} = \frac{28}{30}
      	\]
      	\[
      	1 - \left(\frac{28}{30}\right)^6 = 1 - 0.661029 = 0.338971
      	\]
      \item 
      	Where there are only 3 samples the probabality is
      	\[
      	1 - \left(\frac{28}{30}\right)^3 = 1 - 0.813037 = 0.186963 
      	\]
  \end{enumerate}
  \item[5.4.4]
  We can construct a confidence interval for the sample 
  \[
  	\frac{19 - 20}{\frac{10}{\sqrt{16}}} = -0.4
  \]
  Then the probabality that the estimator will be between 19 and 21 can be
  computed by 
  \[
  	P(19<\bar{X}<21) = P(-0.4 < Z < 0.4) = 0.655422 - 0.344578 = 0.310843
  \]
  \item[5.4.5]
  Theorem 4.2.2 states that for a random variable X with Poisson distribution,
  $E(X) = \lambda$. Therefore the $E(\bar{X})$ of k samples will be
  $\frac{1}{k}\displaystyle\sum_{n=1}^{k} E(X)$ , it follows that the estimator
  $\hat{X} = \frac{kE(X)}{k} = E(X) = \lambda$ and is therefore unbiased. 
  
  Anytime the expected value of a parameter equals the expected value of
  probabality function the parameter is in, the sample mean will be an unbiased
  estimator. 
\end{enumerate}

\end{document}
