\documentclass[11pt,fleqn]{article}
\usepackage{latexsym}
\usepackage{amsmath,amsthm}

\begin{document}
\newcommand{\mbf}[1]{\mbox{{\bfseries #1}}}
\newcommand{\N}{\mbf{N}}
\renewcommand{\O}{\mbf{O}}
\newcommand{\tabfrac}[2]{%
	\setlength{\fboxrule}{0pt}%
	\fbox{$\frac{#1}{#2}$}%
}

\noindent Bill Davis 

\noindent November 5th 2009 

\noindent Homework 9

\begin{enumerate}
  
  \item[9.2.5]
  The two sample variances are 
  \[
  	S_{carpeted}^2 = \frac{8(1053.70) - 89.6^2}{8(7)} = 7.168571429
  \]
    \[
  	S_{non-carpeted}^2 = \frac{8(838.49) - 78.3^2}{8(7)} = 10.304107143
  \]
  The combined variance is
  \[
  	\S_p =\sqrt{ \frac{7(7.168) + 7(10.304)}{8+8-2} } = 2.9556
  \]
  The t-score is 
  \[
  \frac{11.2-9.787}{2.9556\sqrt{\frac{1}{8}+\frac{1}{8}}} =  0.95615
  \]
  $T_{5,14} = 1.7613 > 0.956$, this sample does not imply that the means are
  different and we should not reject $H_0$.
  
  \item[9.2.6]
  \[
  S_p = \sqrt{\frac{30(1.469)^2+56(1.350)^2}{31+57-2}} = 1.39266
  \]
  The t-score is 
  \[
\frac{3.10-2.43}{1.39266\sqrt{\frac{1}{31}+\frac{1}{57}}} = 2.1557
  \]
    $T_{5,86} = 1.6288 < 2.1557$ we reject $H_0$ and say that the means are
  different
  \item[9.2.13]
In order to show that $S_p$ is an unbiased estimator we need to compute the
expected value of $S_p$. 
\[
E(S_p) = \frac{(n-1)\sigma_x^2 + (m-1)\sigma_y^2}{n+m-2}
\]
where $E(S_x) = \sigma_x$ and $E(S_y) = \sigma_y$, and since $\sigma_x =
\sigma_y$
\[
\frac{(n-1)\sigma_x^2 + (m-1)\sigma^2}{n+m-2} =\frac{(n-1 +
m-1)\sigma^2}{n+m-2}  = \sigma^2
\]
Therefore $S_p$ is an unbiased estimator. 

  \item[9.2.15]
  \[
  t = \frac{545.45 - 241.81}{\sqrt{\frac{428^2}{11}+\frac{183^2}{11}}} =
  \frac{303.64}{140.34} = 2.1636
  \]
  To find the degrees of freedom
  \[
  \frac{(\frac{428^2}{11}+\frac{183^2}{11})^2}{\frac{\frac{428}{11}^2}{10} +
  \frac{\frac{183}{11}^2}{10}} = \frac{387993296}{28659414} = 13.53
  \]

	For a one-sided interval $t_{13,0.5}  = 1.7709 < 2.1636$. We therefore reject
	$H_0$, the data does support different means for the two samples. 
  \item[9.5.2]
  In this problem $\mu_x = 6.7$ and $\mu_y = 5.6$. Then $\mu_x-\mu_y = 1.1$.
  and 
  \[
  s_p = \sqrt{\frac{9(0.54)^2+7(0.36)^2}{9+7-2}} = 0.5022
  \]
  And $T_{12,0.5} = 1.7823$ The interval is then plus/minus 
  \[
1.7823(0.5022)\sqrt{\frac{1}{9}+\frac{1}{7}} = 0.4510  
  \]
  The interval is then (0.065,1.55). Since this interval does not include 0
  the data does imply a difference in the silver content of the two samples. 
  
  \item[9.5.3]
  In this problem $\mu_x = 83.96$ and $\mu_y = 84.84$. Then $\mu_x - \mu_y =
  -0.88$. Where $T_{10,2.5} = 2.2281$. The interval will then be plus/minus 
  \[
  2.2281(11.2)\sqrt{\frac{1}{5} + \frac{1}{7}} = 14.611
  \]
  The interval is then (-15.49, 13.73). Which includes 0, therefore the data
  does not suggest that there is a difference between the two groups. 
\end{enumerate}

\end{document}