\documentclass[11pt,fleqn]{article}
\usepackage{latexsym}
\usepackage{amsmath,amsthm}

\begin{document}
\newcommand{\mbf}[1]{\mbox{{\bfseries #1}}}
\newcommand{\N}{\mbf{N}}
\renewcommand{\O}{\mbf{O}}
\newcommand{\tabfrac}[2]{%
	\setlength{\fboxrule}{0pt}%
	\fbox{$\frac{#1}{#2}$}%
}

\noindent Bill Davis 

\noindent September 17th 2009 

\noindent Homework 2

\begin{enumerate} 

\item[2.5.1]
\begin{enumerate}
\item 
Since $P(A \cap B) \neq 0 $, $A$ and $B$ are not mutually exclusive.
\item
Since $P(A) \cdot P(B) = 0.12$, $A$ and $B$ are not independent.
\item
$P(A^{C} \cup B^{C}) = 1 - P(A \cap B) = 1-0.2 = 0.8$

\end{enumerate}

\item[2.5.4]
If the event that both chips have same color is labeled $A$ then
\begin{eqnarray}
 P(A) &=& P(R_1 \cap R_2) + P(B_1 \cap B_2) + P(W_1 
\cap W_2) \\
&=& \frac{3}{10} \cdot \frac{2}{9} + \frac{2}{10} \cdot \frac{4}{9} +
\frac{5}{10} \cdot \frac{3}{9}\\
&=&\frac{29}{90}
\end{eqnarray}

\item[2.5.7]
\begin{enumerate}
  \item
  \begin{enumerate}
      \item 
      If $A$ and $B$ are mutually exclusive then $P(A \cup B) =
      \frac{1}{4} + \frac{1}{8} = \frac{3}{8}$
      \item 
      If $A$ and $B$ are independant then $P(A \cup B) = \frac{1}{4} +
      \frac{1}{8} - \frac{1}{4} \cdot \frac{1}{8} = \frac{11}{32}$
  \end{enumerate}
  \item
  \begin{enumerate}
      \item 
      If $A$ and $B$ are mutually exclusive then $P(A \mid B) =
      \dfrac{0}{\frac{1}{8}} = 0 $
      \item
      If $A$ and $B$ are independent then $P(A \mid B) =
      \dfrac{\frac{1}{32}}{\frac{1}{8}} = \frac{8}{32}$
  \end{enumerate}
\end{enumerate}

\item[3.2.4] 
\begin{enumerate}
  \item
Probabality that 4 will need rework 
\[
{ 12 \choose 4 }(0.15)^4 \cdot (0.85)^{8} = 0.0682844
\]
\item
Probabality that at least 1 needs rework = 
\[
1-P(All Work) = 1- 0.85^{12} = 1 - 0.1422 = 0.8578
\]
\end{enumerate}
\item[3.2.5]
Probabality of being able to work = probabality that 0, 1, or 2 machines are
broken. 
\[
\sum_{n=0}^{2} {10 \choose n}(0.05)^n(0.95)^{10-n} = 0.598 + 0.315 + 0.0746 =
0.9885
\]
\item[3.2.18]
Probabality of choosing a 5 person committee with 2 accountants is 
\[
\frac{ {4 \choose 2} \cdot {8 \choose 3} } { {12 \choose 5} } = \frac{14}{33} =
0.\bar{42}
\]

\item[3.2.25]
Probability that the last gem stolen is the second real diamond, is the same as
saying that in the first three there is one real diamond times the P(last one
is a diamond $\mid$ 1 real and 2 fakes are stolen)

\[
\frac{ {10 \choose 1 } \cdot {25 \choose 2} } {{35 \choose 3}} \cdot
\frac{10-1}{35-3} = \frac{600}{1309} \cdot \frac{9}{32} = \frac{675}{5236} =
0.1289
\]
\item[3.3.5]
\begin{tabular}{l c r | c}
Coin 1 & Coin 2 & Coin 3 & Head - Tails \\
\hline
H & H & H & 3 \\
H & H & T & 1 \\
H & T & H & 1 \\
H & T & T & -1 \\
T & H & H & 1 \\
T & H & T & -1 \\
T & T & H & -1 \\
T & T & T & -3 \\
\hline
\end{tabular} 

Therefore the probabality distribution is 

\begin{tabular}{c c} 
Heads - Tails & P(k) \\
\hline
3 & $\tabfrac{1}{8}$ \\
1 & $\tabfrac{3}{8}$ \\
-1 & $\tabfrac{3}{8}$ \\ 
-3 & $\tabfrac{1}{8}$\\
\hline
\end{tabular}

\item[3.3.6]
The table below shows the sum of the two dice when rolled 

\begin{tabular}{c | c c c c c c |}
 & 1 & 3 & 4  & 5 & 6 & 8 \\
 \hline
 1 & 2 & 4 & 5 & 6 & 7 & 9 \\
 2 & 3 & 5 & 6 & 7 & 8 & 10 \\
 2 & 3 & 5 & 6 & 7 & 8 & 10 \\
 3 & 4 & 6 & 7 & 8 & 9 & 11 \\
 3 & 4 & 6 & 7 & 8 & 9 & 11 \\
 4 & 5 & 7 & 8 & 9 & 10 & 12 \\
 \hline
\end{tabular}

Where the table for the sum of two ordinary dice is 

\begin{tabular}{c | c c c c c c |}
 & 1 & 2 & 3  & 4 & 5 & 6 \\
 \hline
 1 & 2 & 3 & 4 & 5 & 6 & 7 \\
 2 & 3 & 4 & 5 & 6 & 7 & 8 \\
 3 & 4 & 5 & 6 & 7 & 8 & 9 \\
 4 & 5 & 6 & 7 & 8 & 9 & 10 \\
 5 & 6 & 7 & 8 & 9 & 10 & 11 \\
 6 & 7 & 8 & 9 & 10 & 11 & 12 \\
 \hline
\end{tabular}


The probabality distribution derived from both of these tables is the same
and is shown below

\begin{tabular} {c c}
Sum Of Dice & P(sum) \\
\hline
2 & $\tabfrac{1}{36}$ \\
3 & $\tabfrac{2}{36}$ \\
4 & $\tabfrac{3}{36}$ \\
5 & $\tabfrac{4}{36}$ \\
6 & $\tabfrac{5}{36}$ \\
7 & $\tabfrac{6}{36}$ \\
8 & $\tabfrac{5}{36}$ \\
9 & $\tabfrac{4}{36}$ \\
10 &$ \tabfrac{3}{36}$ \\
11 &$ \tabfrac{2}{36}$ \\
12 &$ \tabfrac{1}{36}$ \\
\end{tabular}

\item[3.3.11]
Using the equation for binomial probabality the PDF of X is shown below

\begin{tabular} {c c}
X & P(x) \\
\hline
0 & $\tabfrac{16}{81}$ \\
1 & $\tabfrac{32}{81}$ \\
2 & $\tabfrac{24}{81}$ \\
3 & $\tabfrac{8}{81}$ \\
4 & $\tabfrac{1}{81}$ \\
\end{tabular}

It follows that the PDF of 2X+1 is

\begin{tabular} {c c}
X & P(x) \\
\hline
1 & $\tabfrac{16}{81}$ \\
3 & $\tabfrac{32}{81}$ \\
5 & $\tabfrac{24}{81}$ \\
7 & $\tabfrac{8}{81}$ \\
9 & $\tabfrac{1}{81}$ \\
\end{tabular}
 
and 0 elsewhere. 

\end{enumerate}
\end{document}