\documentclass[11pt,fleqn]{article}
\usepackage{latexsym}
\usepackage{amsmath,amsthm}

\begin{document}
\newcommand{\mbf}[1]{\mbox{{\bfseries #1}}}
\newcommand{\N}{\mbf{N}}
\renewcommand{\O}{\mbf{O}}
\newcommand{\tabfrac}[2]{%
	\setlength{\fboxrule}{0pt}%
	\fbox{$\frac{#1}{#2}$}%
}

\noindent Bill Davis 

\noindent September 24th 2009 

\noindent Homework 3

\begin{enumerate} 
  
\item[3.6.14]
$f_Y(y) = (3(1-y)^2) $
\[
E(Y) = \int_0^1 3y(1-y)^2 \,dy =1.5y^2 - 2y^3 + .75y^4 |_0^1  = 1.5-2+.75 =
0.25
\]
\[ 
E(Y^2) = \int_0^1 3y^2(1-y)^2 \,dy =y^3-1.5y^4+0.6y^5 |_0^1 = 1-1.5+.6 = .1
\]
\[
E(Y^2) - E^2(Y) = \frac{1}{10} - \frac{1}{16} = \frac{3}{80}
\]

Then the variance of W if W = $-5Y+12$ is $(-5)^2(\frac{3}{80}) = \frac{75}{80}$
\item[3.7.1]

If $p_{X,Y}(x,y) = cxy$ at points (1,1),(2,1),(2,2),(3,1) then $p(1,1) = c$ and
$p(2,1) = 2c$ and $p(2,2) = 4c$ and $p(3,1) = c$. In which case 
\[
c + 2c + 3c + 4c = 1 = 10c
\] 

It follows that $c=\frac{1}{10}$.

\item[3.7.2]
\[
\int_{-1}^1 \int_{-1}^1 cx^2+cy^2\,dy\,dx = \int_{-1}^{1} 2cx^2 +
\frac{2}{3}{c} \, dx = \frac{2}{3}cx + \frac{2}{3}cx^3 \, |_0^1 = \frac{8}{3}c
\]

And since $\frac{8}{3}c = 1$ then	 $c = \frac{3}{8}$.

\item[3.7.10]
\begin{enumerate}
  \item
	In this case 
	\[
	\int_0^1\int_0^1 c \,dy\,dx = c
	\]
	Therefore $c=1$
  \item
  \[
  	\int_0^{0.5}\int_0^{0.25} 1 \,dy\,dx = \int_0^{0.5} 0.25 \,dx = 0.125
 \]
\end{enumerate}
\item[3.7.20]

\begin{enumerate}
  \item
\[
f_{X,Y}(x,y) = \frac{1}{2}
\]
\[
f_X(x) = \int_0,2 \frac{1}{2}\,dy = \frac{1}{2}y |_0^2 = 1
\]
\[
f_Y(y) = \int_0,2 \frac{1}{2}\,dx = \frac{1}{2}x |_0^2 = 1
\]

\items
\[
f_{X,Y}(x,y) =  \frac{1}{x}
\]
\[
f_X(x) = \int_0,1 \frac{1}{x}\,dy = \frac{1}{x}y |_0^1 = \frac{2}{x}
\]
\[
f_Y(y) = \int_0,1 \frac{1}{x}\,dx = ln(x)|_0^2 = 1
\]
\end{enumerate}

\item[4.2.3]
The probabality that at most 1 person was born on Poisson's birthday is
\[
\sum_n=0^1 \frac{e^{\frac{-500n}{365}}(\frac{n}{365})^k} {0!}
\]

Where n=0 
\[
e^{ \frac{-500}{365}  } = 0.25414
\]

Where n=1
\[
\frac{ e^{\frac{-500}{365}}(\frac{500}{365})  }{ 1!  } = 0.3481
\]
 
So, $0.25414 + 0.3481 = 0.60224$.

\item[4.2.21]
\begin{enumerate}

\item
The probabality that there will be 2 accidents is

\[
\frac{e^{0.5}(0.5)^{2}}{2!} = 0.2061
\]
  
\item
The probabality that there will be 4 accidents over the next 10 days is
\[
\frac{e^1(1)^4}{4!} = 0.11326
\]

\end{enumerate}


\end{enumerate}

\end{document}
