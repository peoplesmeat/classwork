\documentclass[11pt,fleqn]{article}
\usepackage{latexsym}
\usepackage{amsmath,amsthm}

\begin{document}
\newcommand{\mbf}[1]{\mbox{{\bfseries #1}}}
\newcommand{\N}{\mbf{N}}
\renewcommand{\O}{\mbf{O}}
\newcommand{\tabfrac}[2]{%
	\setlength{\fboxrule}{0pt}%
	\fbox{$\frac{#1}{#2}$}%
}

\noindent Bill Davis 

\noindent October 22nd 2009 

\noindent Homework 7

\begin{enumerate}

\item[6.4.3]
To compute $\beta$ we need to compute the bounds for $H_0$ at $\alpha=0.06$
assuming a two sided bound. This means we will reject $H_0$ if the sample mean
is less then 88.9852 or greater then 101.015. The Z score for a sample
mean being greater then 88.9852 if the the mean is 90 is 
\[
\frac{88.9852 - 90}{\frac{15}{\sqrt{22}}} = -0.317322. 
\]
Which corresponds to a probabality of 0.6245. Then 1 - $\beta$ is 0.3755.
\item[6.4.5]
If $H_0$ is that $\mu=240$ at 0.01 with 25 samples, $H_0$ is rejected if the
sample mean is below 216.737. The z score for 216.737 if the sample mean is 220
\[
\frac{216.737 - 220.0}{\frac{50}{\sqrt{25}}} = -0.3263. 
\]
Then the probabality of not detecting that the mean has reduced to 220
corresponds to a probabality of 0.6279.
\item[7.3.4]
Since the variance of a chi-square random variable with k degrees of freedom is
2k, 
\[
\mathrm{Var}(\frac{(n-1)S^2}{\sigma^2}) = 2(n-1)
\]
and since $\mathrm{Var}(aw) = a^2\mathrm{Var}(w)$
\[
\mathrm{Var}(\frac{(n-1)S^2}{\sigma^2}) = \frac{(n-1)^2
\mathrm{Var}(S^2)}{\sigma^4}
\]
Then 
\[
\frac{(n-1)^2 \mathrm{Var}(S^2)}{\sigma^4}= 2(n-1) 
\]
and solving for $\mathrm{Var}(S^2)$
\[
\mathrm{Var}(S^2) = \frac{2\sigma^4}{n-1}
\]
\item[7.3.9]
\begin{enumerate}
  \item $F^{-1}_{4,6}(0.975) = 6.23$
  DOOOOOOOOOOOOOOOOOOOOOOOOOOOOOOOOOOMMMMMMMMMMMMMMMMMOOOOOOOOOOOOOOOOOORRRRRRRR
\end{enumerate}
\item[7.4.1]
\begin{enumerate}
  \item 0.15
  \item 1-0.20.80
  \item 0.85
  \item 0.99-0.15= 0.84
\end{enumerate}
\item[7.4.7]
The P(T>3.24984) = 0.995, which will give us a 99 percent confidence interval.
We need to solve the equation 
\[
\frac{x-0.484}{\frac{0.239}{\sqrt{10}}} = 3.24984
\]
X = 0.729617, therefore the 99\% interval is $0.246<X<0.729$.
\item[7.4.16]
This would not be reasonable, it doesn't make sense to extrapolate from these
cities to all of the United States. For example geographic distribution and
population size play a huge role in median home prices and there is no
evidence that these cities represent a good cross section of these parameters.
There are also only 16 cities, which won't even provide a city for each state.
At the minimum the population of these cities needs to be taken into account
when constructing a median home resale value. 
\item[7.5.3]
\begin{enumerate}
  \item 2.088
  \item 7.261
  \item 14.041
  \item 17.539
\end{enumerate}
\item[7.5.9]
The sample variance for this problem is 26.4860. 
\begin{enumerate}
  \item
  The 95\% confidence interval for 18-1 degrees of freedom is 7.564, 30.191. 
  $\sqrt{\displaystyle\frac{17(26.4860)}{30.191}} = 3.8619$ and
  $\sqrt{\displaystyle\frac{17(26.4860)}{7.564}} = 7.7155$. 
  
  The bounds are therefore (3.862, 7.16). 
  \item
  Two one-sided bounds are  
  $\sqrt{\displaystyle\frac{17(26.4860)}{27.587}} = 4.040$ and
  $\sqrt{\displaystyle\frac{17(26.4860)}{8.672}} = 7.205$
  
  The bounds are therefore X<7.205 and X>4.040. 
\end{enumerate}
\item[7.5.15] 

 
\end{enumerate}

\end{document}
