\documentclass[11pt,fleqn]{article}
\usepackage{latexsym,epsf,epsfig}
\usepackage{amsmath,amsthm}
\usepackage{xy}
\input xy
\xyoption{all}
\begin{document}
\newcommand{\mbf}[1]{\mbox{{\bfseries #1}}}
\newcommand{\N}{\mbf{N}}
\renewcommand{\O}{\mbf{O}}

\noindent Bill Davis \\
\noindent 605.411 Problem Set 7
\begin{enumerate}
\item %Problem 1
  \begin{enumerate}
  \item If the cache was direct mapped it would need a 2 bit tag. 
  \item   \begin{tabular}{|r|r|r|}
  \hline
	Line & Tag & Data \\
  \hline
   0 & 1 1 & Data from line 12 \\
   1 & 0 1 & Data from line 5 \\
   2 & 1 1 & Data from line 14 \\
   3 & 0 1 & Data from line 7 \\
  \hline
  \end{tabular}
  \end{enumerate}
\item %Problem 2
This cache contains 32 sets with 4 ways. \\
\begin{enumerate}
\item 
\begin{tabular}{r|r}
\hline
293824& Maps to set 7, changes the set 7 LRU bits to 110 \\
2948  & Maps to set 0, changes the set 0 LRU bits to 110 \\
41728 & Maps to set 10, changes the set 10 LRU bits to 110 \\
3072  & Maps to set 0, this line is already in the cache, no changes to LRU \\
1920  & Maps to set 0, this line is already in the cache, no changes to LRU \\
5016  & Maps to set 1, changes the set 1 LRU bits to 110 \\
6536  & Maps to set 1, this line is already in the cache, no changes to LRU \\
293764& Maps to set 7, this line is already in the cache, no changes to LRU \\
4088  & Maps to set 0, this line is already in the cache, no changes to LRU \\
3184  & Maps to set 0, this line is already in the cache, no changes to LRU \\
\hline
\end{tabular}
\item In this instance all replacements are correct, while in theory this pseudo LRU replacement scheme could identify a line which is not actually the least recently used. 
\end{enumerate}
\item %Problem 3
See attached sheet
\item %Problem 4
Since C uses row major order, the array will be stored in memory as follows \\
0x3EA4000: (0,0) (0,1) (0,2) ... (0,31) (1,0) ... (1,31) ... (31,31) \\
0x3EA5000: (32,0) (32,1) ... (32,31) (33,0) ... (33,31) ... (63,31)\\
0x3EA6000: (64,0) (64,1) ... (64,31) (65,0) ... (65,31) ... (95,31)\\
\\
Where each 4096 byte page can hold 32 consecutive rows from the array. Also since we 32768 bytes available in main memory, we could potentially hold 8 pages or half the entire array in memory at any given time. 
\begin{enumerate}
\item %Problem 4a 
Here the memory access follow the pattern \\
(0,0) (1,0) (2,0) ... (0,1) (0,2) ... (348,29) (349,29) \\
This means there is 1 page fault very 32 accesses. Also, since there are only 8 pages in memory at any given time, by the time we have accessed row 349 and are ready to wrap around back to 0, row 0 has been paged out to disk. \\
\\
This means there are (11 page faults per column for 30 columns) = 330 page faults
\item %Probelm 4b
Here the memory accesses follow the pattern \\
(0,0) (0,1) (0,2) (0,3) ... (1,0) (1,1) ... (349, 28) (349,29) \\
This means there are only 11 page faults, since we can read every column in a row off the same page.
\end{enumerate} 
\item %Problem 5
Yes, a program can determine whether it is running on a big-endian machine by storing an integer and checked whether it ended up in the high-order byte or not. \\
addui \$1, \$0, 1 \\
sw \$1 0(\$2) //Store word to address\\ 
lb \$1 0(\$2)  //Load in first byte \\
andi \$t0 \$1 1  //If first byte equals 1, then big endian \\


\end{enumerate}
\end{document}
