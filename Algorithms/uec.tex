\documentstyle[12pt]{article}

\input amssymb.sty
%\input amssym.def
%\input amssym.tex


\newcommand{\arrows}{\longrightarrow}
\newcommand{\initial}{{\vartrianglelefteq}}

\newcommand{\seq}[1]{\buildrel\rightharpoonup\over{#1}}
\newcommand{\eop}{\bigstar}  % end-of-proof
\newcommand{\kap}{K_{\rm ap}}
\newcommand{\complB}{<\!\!\!\circ}
\newcommand{\card}[1]{{\vert #1 \vert} }
\newcommand{\norm}[1]{{\card{\card{#1}}}}
\newcommand{\otp}[1]{\hbox{otp($#1$)}}
\newcommand{\forces}{\Vdash}
\newcommand{\decides}{\parallel}
\newcommand{\ndecides}{\nparallel}
%\newcommand{\models}{\vDash}
\newcommand{\lessB}{<\!\!\!\circ}

\newcommand{\ap}{{\rm ap}}
\newcommand{\dom}{{\rm dom}}
\newcommand{\Dom}{{\rm Dom}}
\newcommand{\Ev}{{\rm Ev}}
\newcommand{\id}{{\rm id}}
\newcommand{\llg}{{l\rm g}}
\newcommand{\Rang}{{\rm Rang}}
\newcommand{\rge}{{\rm rge}}
\newcommand{\crit}{{\rm crit}}
\newcommand{\supp}{{\rm supp}}
\newcommand{\support}{{\rm support}}
\newcommand{\cf}{{\rm cf}}
\newcommand{\lenght}{{\rm lg}}
\newcommand{\md}{{\rm md}}


\newcommand{\implies}{\Longrightarrow}
\newcommand{\vtl}{\vartriangleleft}

\newenvironment{proof}{\noindent{\bf Proof.}}{\par\bigskip}

\newenvironment{Proof}{\noindent{\bf Proof.}}{\par\bigskip} 

\newtheorem{THEOREM}{Theorem}[section]
\newenvironment{theorems}{\begin{THEOREM}}
{\end{THEOREM}}

\newtheorem{Conclusion}[THEOREM]{Conclusion}
\newenvironment{conclusion}{\begin{Conclusion}}{\end{Conclusion}}

\newtheorem{LEMMA}[THEOREM]{Lemma}
\newenvironment{lemmas}{\begin{LEMMA}}{\end{LEMMA}}

\newtheorem{Main Theorem}[THEOREM]{Main Theorem}
\newenvironment{main Theorem}{\begin{Main Theorem}} 
{\end{Main Theorem}}

\newtheorem{Theorem}[THEOREM]{Theorem}
\newenvironment{theorem}{\begin{Theorem}}{\end{Theorem}}

\newtheorem{Definition}[THEOREM]{Definition}
\newenvironment{definition}{\begin{Definition}}{\end{Definition}}

\newtheorem{Conventions}[THEOREM]{Conventions}
\newenvironment{conventions}{\begin{Conventions}}{\end{Conventions}}


\newtheorem{Main Definition}[THEOREM]{Main Definition}
\newenvironment{main definition}{\begin{Main Definition}}
{\end{Main Definition}}

\newtheorem{Lemma}[THEOREM]{Lemma}
\newenvironment{lemma}{\begin{Lemma}}{\end{Lemma}}

\newtheorem{Notation}[THEOREM]{Notation}
\newenvironment{notation}{\begin{Notation}}{\end{Notation}}

\newtheorem{Convention}[THEOREM]{Convention}
\newenvironment{convention}{\begin{Convention}}{\end{Convention}}

\newtheorem{Note}[THEOREM]{Note}
\newenvironment{note}{\begin{Note}}{\end{Note}}

\newtheorem{Observation}[THEOREM]{Observation}
\newenvironment{observation}{\begin{Observation}}
{\end{Observation}}

\newtheorem{Remark}[THEOREM]{Remark}
\newenvironment{remark}{\begin{Remark}}{\end{Remark}}

\newtheorem{Main Fact}[THEOREM]{Main Fact}
\newenvironment{main Fact}{\begin{Main Fact}}{\end{Main Fact}}

\newtheorem{Fact}[THEOREM]{Fact}
\newenvironment{fact}{\begin{Fact}}{\end{Fact}}

\newtheorem{Subfact}[THEOREM]{Subfact}
\newenvironment{subfact}{\begin{Subfact}}{\end{Subfact}}

\newtheorem{Claim}[THEOREM]{Claim}
\newenvironment{claim}{\begin{Claim}}{\end{Claim}}

\newtheorem{Main Claim}[THEOREM]{Main Claim}
\newenvironment{main claim}{\begin{Main Claim}}{\end{Main Claim}}

\newtheorem{Corrolary}[THEOREM]{Corrolary}
\newenvironment{corrolary}{\begin{Corrolary}} 
{\end{Corrolary}}

\newtheorem{Subclaim}[THEOREM]{Subclaim}
\newenvironment{subclaim}{\begin{Subclaim}}{\end{Subclaim}}

\newtheorem{Corollary}[THEOREM]{Corollary}
\newenvironment{corollary}{\begin{Corollary}}{\end{Corollary}}

\newtheorem{Example}[THEOREM]{Example}
\newenvironment{example}{\begin{Example}}{\end{Example}}


\newtheorem{Proposition}[THEOREM]{Proposition}
\newenvironment{proposition}{\begin{Proposition}}{\end{Proposition}}

\newtheorem{Discussion}[THEOREM]{Discussion}
\newenvironment{discussion}{\begin{Discussion}}{\end{Discussion}}

%{\nopagebreak\mbox{}\newline
%\makebox[\textwidth]{\hfill\eop}\par\bigskip}

\newenvironment{Proof of the Subfact}
{\noindent{\bf Proof of the Subfact.}}{\par\bigskip}

\newenvironment{Proof of the Theorem}
{\noindent{\bf Proof of the Theorem.}}{\par\bigskip}

\newenvironment{Proof of the Conclusion}
{\noindent{\bf Proof of the Conclusion.}}{\par\bigskip}

\newenvironment{Proof of the Observation}
{\noindent{\bf Proof of the Observation.}}{\par\bigskip}

\newenvironment{Proof of the Fact}
{\noindent{\bf Proof of the Fact.}}{\par\bigskip}

\newenvironment{Proof of the Lemma}
{\noindent{\bf Proof of the Lemma.}}{\par\bigskip}

\newenvironment{Proof of the Claim}
{\noindent{\bf Proof of the Claim.}}{\par\bigskip}

\newenvironment{Proof of the Subclaim}
{\noindent{\bf Proof of the Subclaim.}}{\par\medskip}

\newenvironment{Proof of the Main Claim}
{\noindent{\bf Proof of the Main Claim.}}{\par\bigskip}

%{\nopagebreak\mbox{}\newline
%\makebox[\textwidth]{\hfill\eop}\par\bigskip}

% 16 April: changed font in environments made by newtheorem to roman
% (Kids don't try this at home!)
% 20 April: put . after theorem number
\catcode`\@=11
\def\@begintheorem#1#2{\rm \trivlist \item[\hskip \labelsep{\bf #1\ #2.}]}
\def\@opargbegintheorem#1#2#3{\rm \trivlist
      \item[\hskip \labelsep{\bf #1\ #2\ (#3).}]}
\catcode`\@=12


\newcommand{\andm}{{\,\,\&\,\,}}

\newcommand{\stick}
{\mbox{{\hspace{0.4ex}$\hbox{\raisebox{1ex}{$\bullet$}}
\hspace*{-0.94ex}|$\hspace{0.6ex}}}}

\newcommand{\height}{{\rm ht}}
\newcommand{\lev}{{\rm lev}}
\newcommand{\Suc}{{\rm Suc}}
\newcommand{\pr}{{\rm pr}}
\newcommand{\apr}{{\rm apr}}

% more fonts
\font\bigbold = cmb10 scaled \magstep2
\font\bigtenrm = cmr10 scaled \magstep1

\newcommand{\elementary}{\prec}

\newcommand{\Bbf}{\Bbb}
\newcommand{\vartrianglelefteq}{\trianglelefteq}
\newcommand{\bwedge}{\bigwedge}
\newcommand{\cwedge}{\bigwedge}
\newcommand{\bvee}{\bigvee}
\newcommand{\symmdif}{\bigtriangleup}
\newcommand{\tensor}{\otimes}

\newcommand{\TV}{{\rm TV}}
\newcommand{\Club}{{\rm Club}}
\newcommand{\Con}{{\rm Con}}
\newcommand{\Max}{{\rm Max}}
\newcommand{\Sk}{{\rm Sk}}

\newcommand{\order}{\prec}
\newcommand{\club}{\clubsuit}
\newcommand{\bd}{{\rm bd}}                      %bounded ideal
\newcommand{\into}{\rightarrow}

\newcommand{\rest}{\upharpoonright}  % restriction
\newcommand{\for}{\forall}
\newcommand{\nacc}{\mathop{\rm nacc}}
\newcommand{\acc}{\mathop{\rm acc}}
\newcommand{\cl}{\mathop{\rm cl}}
\newcommand{\cov}{\mathop{\rm cov}}
\newcommand{\gl}{\mathop{\rm gl}}
\newcommand{\scl}{\mathop{\rm scl}}
\newcommand{\satisfies}{\vDash}
\newcommand{\deq}{\buildrel{\rm def}\over =}
\newcommand{\Min}{\mathop{\rm Min}}% minimum
\newcommand{\mod}{\mathop{\rm mod}}

% caligraphic letters
\newcommand{\AAA}{{\cal A}}
\newcommand{\BB}{{\cal B}}
\newcommand{\CC}{{\cal C}}
\newcommand{\DD}{{\cal D}}
\newcommand{\EE}{{\cal E}}
\newcommand{\FF}{{\cal F}}
\newcommand{\GG}{{\cal G}}
\newcommand{\HH}{{\cal H}}
\newcommand{\II}{{\cal I}}
\newcommand{\JJ}{{\cal J}}
\newcommand{\KK}{{\cal K}}
\newcommand{\PP}{{\cal P}}
\newcommand{\TT}{{\cal T}}

% projection functions
\newcommand{\Piba}{\Pi^\beta_\alpha}
\newcommand{\piba}{\pi_\alpha^\beta}  % from 2^beta to 2^alpha

%\input name.tex
\title{On uniform Eberlein compacta and c-algebras}
\author{Mirna D\v zamonja\\
School of Mathematics\\
University of East Anglia\\
Norwich, NR4 7TJ, UK\\
\scriptsize{M.Dzamonja@uea.ac.uk}}

\begin{document}


\baselineskip=16pt
\binoppenalty=10000
\relpenalty=10000
\raggedbottom

\maketitle

\begin{abstract} We investigate a question of Y. Benyamini, M.~E. Rudin and
M. Wage, on the existence of universal uniform Eberlein compacta of a
given weight, more exactly the related question of the
existence of a universal c-algebra of a given size. We show that for any
regular $\lambda>\aleph_1$ with
$2^{\aleph_0}>\lambda$, there is no
c-algebra of size $\lambda$ universal under c-embeddings.
In fact, under these circumstances, for no $\mu<2^{\aleph_0}$
are there $\mu$ c-algebras of size $\lambda$ such that
every c-algebra of size $\lambda$ c-embeds into
one of the $\mu$ given ones.
{\footnote{The author thanks Saharon Shelah for bringing M. Bells' \cite{Bell}
to her attention.
This paper is dedicated to the memory of Amer Be\v slagi\'c.}
}
\end{abstract} 

\section{Introduction.}
We investigate the question of the existence of
universal uniform Eberlein compacta. 
The question of the existence of universal uniform Eberlein
compacta of a given size was asked by Y. Benyamini, M.E. Rudin and
M. Wage as Problem 3 in their 1977 paper \cite{Rudin}.
In his recent paper \cite{Bell},
M. Bell showed that this is equivalent to the existence of universal
c-algebras under ordinary embeddings. He showed in \cite{Bell}
that if $\lambda=2^{<\lambda}$, then there is a c-algebra
of size $\lambda$ which is universal not just under ordinary embeddings,
but also under a stronger notion of a c-embedding.
We are interested in the question of the existence of such an algebra
when the relevant instances of $GCH$ fail.
We show that for any  regular cardinal $\lambda>\aleph_1$
with $2^{\aleph_0}>\lambda$, there can be no c-algebra
of size $\lambda$ into which every c-algebra of size $\lambda$ c-embeds.
In fact, under these circumstances, for no $\mu<2^{\aleph_0}$
are there $\mu$ c-algebras of size $\lambda$ such that every
c-algebra of size $\lambda$ embeds into
one of the $\mu$ given ones.
The niceness property of the c-algebras is not used in the above.
We also show that $\clubsuit + \neg CH$ implies that there is no c-algebra
of size $\aleph_1$ into which every c-algebra of size $\aleph_1$ c-embeds,
in fact, there are no $\mu<2^{\aleph_0}$ c-algebras of
size $\aleph_1$ such that every c-algebra of size $\aleph_1$ c-embeds
into one of them. 
In fact, $\clubsuit_{{\rm club}}$
suffices in the place of $\clubsuit$.
It is a result of M. Bell from \cite{Bell} that adding
$\aleph_2$ Cohen reals suffices to have a model in which there is no
universal uniform Eberlein compact of weight $\aleph_1$, hence also
no c-algebra of size $\aleph_1$ universal even under ordinary embeddings.

A uniform Eberlein compact, abbreviated UEC, is
a topological space homeomorphic to a
weakly compact subspace of a Hilbert space.
A UEC $X^\ast$ is said to be
{\em universal}
of weight $\lambda$ iff every UEC of weight $\le\lambda$ is a continuous image of it.
The intuition suggesting
that a universal object is the one into which all other objects of that
kind embed is justified when one passes to the objects roughly dual to the
UEC, so called c-algebras. Calling a c-algebra $B^\ast$ universal iff every
other c-algebra of the same size embeds into $B^\ast$, it turns out that
there is a universal UEC of weight $\lambda$ iff there is a universal
c-algebra of size $\lambda$ (M. Bell in \cite{Bell}).
Note that this result does not follow immediately from the Stone duality
theorem, as not every UEC is 0-dimensional.

For a history and further references on UEC, see papers
\cite{Rudin} and \cite{Bell}. 
The method
of invariants used in this paper was developed in the context of linear orders
by M. Kojman and S. Shelah in \cite{KjSh 409}, and they have used it in other
contexts elsewhere.


\section{Preliminaries.}\label{basics}

\begin{Notation} For $\alpha>\theta=\cf(\theta)$, we let
\[
S^\alpha_\theta\deq\{\beta<\alpha:\,\cf(\beta)=\theta\}.
\]
\end{Notation}

\begin{Definition}(1) A 
boolean algebra $B$ is a c-{\em algebra} iff there is
a family $\{ B_n:\,n<\omega\}$ of subsets of $B$ such that
\begin{description}
\item{(i)} $n\neq m\implies B_n\cap B_m=\emptyset$,
\item{(ii)} Each $B_n$ consists of pairwise disjoint elements,
\item{(iii)} $\bigcup_{n<\omega} B_n$ generates $B$,
\item{(iv)} $\bigcup_{n<\omega} B_n$ has {\em the nice property}, meaning
that for no finite $F\subseteq \bigcup_{n<\omega} B_n$ do we have $\bigvee F=1$.
\end{description}
We say that $\langle B_n:\,n<\omega\rangle$ witnesses that $B$ is a
c-algebra. When discussing c-algebras, we always have in mind a fixed sequence
witnessing this, although we may omit to mention it. We may refer to it as
$\langle B_n(B):\,n<\omega\rangle$.

\medskip

{\noindent (2)} If $B_l^\ast$ for $l\in \{0,1\}$ are c-algebras,
then a 1-1 boolean 
homomorphism $f:\,B_0^\ast\into B^\ast_1$ is a c-{\em embedding} iff
$f``B_n(B_0^\ast)\subseteq B_n(B_1^\ast)$ for all $n<\omega$. If there is
such an embedding from $B^\ast_0$ to $B^\ast_1$, we write 
\[
B_0^\ast<B^\ast_1.
\]

{\noindent (3)} A c-algebra $B^\ast$
of size $\lambda$ is {\em universal}
$\lambda$ iff for any c-algebra $B$ of size $\le\lambda$, there is an
embedding $f:\,B\into B^\ast$.

{\noindent (4)} A c-algebra $B^\ast$
of size $\lambda$ is {\em c-universal}
$\lambda$ iff for any c-algebra $B$ of size $\le\lambda$, there is a
c-embedding $f:\,B\into B^\ast$.

{\noindent (5)} A boolean algebra is {\em almost c} iff it satisfies
all the properties of c-algebras, except possibly for (iv) in (1) above.

\end{Definition}

\begin{Fact}\label{Bell1} (M. Bell, \cite{Bell}) (1) There is a universal UEC
of size $\lambda$ iff there is a universal c-algebra of size $\lambda$.

{\noindent (2)} If $\lambda=2^{<\lambda}$, then there is a c-universal
c-algebra of size $\lambda$.
\end{Fact}

\begin{Definition} Let $\lambda=\cf(\lambda)>\aleph_0$. 

{\noindent (1)} A sequence $\langle c_\delta:\,\delta\in S\rangle$
is called a {\em club-sequence} iff $S$ is a stationary set of limit ordinals
$<\lambda$, and for each $\delta\in S$, the set $c_\delta$ is a club subset
of $\delta$.

{\noindent (2)} A club sequence $\langle c_\delta:\,\delta\in S\rangle$
is called {\em a club-guessing sequence} iff for every club $E$ of $\lambda$
there is a stationary set of $\delta\in S$ such that $c_\delta
\subseteq E$.
\end{Definition}

The following fact is a result of S. Shelah from \cite{Sh e} [III 7.8], see
also \cite{Sh g}, and a proof is also in M. Kojman and
S. Shelah's \cite{KjSh 409}.

\begin{Fact}\label{guessing} (S. Shelah, \cite{Sh e}) Suppose $\lambda=
\cf(\lambda)>\aleph_1$.

\underline{Then} there is a club guessing sequence
$\langle c_\delta:\,\delta\in S\subseteq S^\lambda_{\aleph_0}\rangle$.
\end{Fact}

\begin{Definition}(1) $\clubsuit$ is the statement that there is a
sequence
\[
\langle c_\delta:\,\delta\mbox{ limit }<\omega_1\rangle
\]
with
$c_\delta\subseteq\delta=\sup(c_\delta)$ such that for every
$A\subseteq \omega_1$ unbounded,
\[
\{\delta:\,c_\delta\subseteq A\}
\]
is stationary.

{\noindent (2)} $\clubsuit_{{\rm club}}$ is the statement obtained
from $\clubsuit$ by replacing above in
``every
$A\subseteq \omega_1$ unbounded", the word ``unbounded" by ``club".
\end{Definition}


\section{Non-existence of c-universal c-algebras.}

\begin{Theorem}\label{theorem} Suppose that $\lambda=\cf(\lambda)>\aleph_1$
satisfies $2^{\aleph_0}>\lambda$. 

\underline{Then} there is no c-universal c-algebra of size $\lambda$,
moreover for no $\mu<2^{\aleph_0}$ are there $\mu$ c-algebras
of size $\lambda$ such that every c-algebra of size
$\le\lambda$ c-embeds into one of them.
\end{Theorem}

\begin{Proof} Fix $\lambda$ as in the hypothesis, and a club guessing
sequence
\[
\bar{c}=\langle c_\delta:\,\delta\in S\subseteq S^\lambda_{\aleph_0}\rangle
\]
as guaranteed by Fact \ref{guessing}.

\begin{Definition}\label{defa}
(1) Suppose that $B$ is a boolean algebra of size $\lambda$.
A sequence $\bar{B}=\langle B^\alpha:\,\alpha<\lambda\rangle$ is a
{\em filtration of} $B$ iff
\begin{description}
\item{(i)} $B^\alpha\subseteq B^{\alpha+1}$ and $B^0=\emptyset$.
\item{(ii)} $B^\delta=\bigcup_{\alpha<\delta} B^\alpha$ for $\delta$ limit,
\item{(iii)} $\card{B^\alpha}<\lambda$,
\item{(iv)} $\bigcup_{\alpha<\lambda} B^\alpha=B$.
\end{description}

{\noindent (2)} Suppose that $B$ is a c-algebra of size $\lambda$ with
a filtration $\bar{B}$, while $\delta\in S$ and
$b\in B\setminus B^\delta$. We define
\[
{\rm Inv}_{\bar{B}}(b,c_\delta)\deq\left\{
\alpha\in c_\delta:(\exists m\ge 1)
(\exists y\in B_m(B)\cap B^{\min(c_\delta\setminus (\alpha+1))}\setminus B^\alpha)
\,[y\ge b]\right\}.
\]
If $\alpha\in {\rm Inv}_{\bar{B}}(b,c_\delta)$ because $m\ge 1$ is such that
for some $y\ge b$ we have $y\in B_m(B)\cap 
B^{\min(c_\delta\setminus (\alpha+1))}\setminus B^\alpha$,
we say
that $\alpha
\in {\rm Inv}_{\bar{B}}(b,c_\delta)$ by virtue of $m$.
\end{Definition}

\begin{Note}\label{observe} With the notation of Definition \ref{defa}:

{\noindent (1)} For every $m\ge 1$, there is at most one $\alpha\in
{\rm Inv}_{\bar{B}}(b,c_\delta)$ which is there by virtue of $m$
(unless $b=0$).

[Why? As elements of $B_m(B)$ are pairwise disjoint.]

{\noindent (2)}
$\card{\{A:\,(\exists b\in B)(\exists\delta\in S) {\rm Inv}_{\bar{B}}(b,c_\delta)
=A\}}\le\lambda$.
\end{Note}

\begin{Main Claim}\label{main} Suppose that
for $\delta\in S$ we are given $A_\delta\subseteq c_\delta$
with $\sup(A_\delta)=\delta$ and $\otp{A_\delta}=\omega$.

\underline{Then} there is a c-algebra $B$ and filtration $\bar{B}$ of $B$ such that
for all $\delta\in S$
there is $b^0_\delta\in B\setminus B^\delta$
such that ${\rm Inv}_{\bar{B}}(b^0_\delta,c_\delta)=A_\delta$.
\end{Main Claim}

\begin{Proof of the Main Claim} Let $f_\delta:\,\omega\into A_\delta$ be the
increasing enumeration of $A_\delta$.
We shall use $\alpha^\delta_m$ to stand for $f_\delta(m)$.

For $n<\omega$, let $X_n$ be the boolean algebra
generated by $\{a_\eta:\,\eta\in {}^{\le n}\lambda\}$ freely except for the
equations
\begin{description}
\item{(1)} $a_{\langle\rangle}=1$,
\item{(2)} $[\eta\neq\varepsilon\,\,\&\,\,\lg(\eta)=\lg(\varepsilon)]
\implies a_\eta\wedge a_\varepsilon=0$,
\item{(2)} $\eta\initial\varepsilon\implies a_\eta\ge a_\varepsilon$.
\end{description}

Let ${\cal B}^\ast$ be the product algebra $\langle
X_n:\, n<\omega\rangle$. Let $I$ be the ideal of eventually 0 elements
of ${\cal B}^\ast$, and let ${\cal B}\deq{\cal B}^\ast/I$. 

For $\delta\in S$ let:
\[
b^0_\delta\deq[\langle a_{f_\delta\rest 0}, a_{f_\delta\rest 1},\ldots
a_{f_\delta\rest n},\ldots\rangle]
\]
and for $m\ge 1$ let $b^m_\delta$ be the class of the function in ${\BB}^\ast$
which is constantly equal to $a_{f_\delta\rest m}$.
Note that $b^m_\delta\ge b^0_\delta$ for all $m\ge 1$.
Also note that $b^0_{\delta}$ are pairwise disjoint, as for $\delta_1\neq\delta_2$
both in $S$, we have that $A_{\delta_1}\cap A_{\delta_2}$ is finite.

For $n<\omega$ let $B_n\deq\{b^n_\delta:\,\delta\in S\}$.
Our algebra $B$ is given by
\[
B\deq\langle \bigcup_{n<\omega}B_n\rangle_{\BB}.
\]
It is easily seen
that $B$ is a c-algebra of size $\lambda$.

Next we define the filtration $\bar{B}=\langle B^\alpha:\,\alpha<\lambda\rangle$
of $B$ by letting $B^0=\emptyset$ and for $\alpha<\lambda$
\[
B^{\alpha+1}\deq\left\langle \bigcup_{n<\omega} B_n\cap\{[\bar{b}]:\,(\forall n)
(\exists \eta\in {}^{\le n}\alpha)\,[(b(n)=a_\eta\mbox{ or } b(n)=0]\}
\right\rangle_B,
\]
while $B^\alpha\deq\bigcup_{\beta<\alpha}B^\beta$ for $\alpha$ limit.

Clearly, $\bar{B}$ is a filtration of $B$, and
$b^0_\delta\in B\setminus B^\delta$. The proof will be finished
once we prove:

\begin{Subclaim}\label{exact}
For $\delta\in S$ we have ${\rm Inv}_{\bar{B}}(b^0_{\delta},
c_\delta)=A_\delta$.
\end{Subclaim}

\begin{Proof of the Subclaim} Note that for $\delta\in S$ and $m\ge 1$
we have
$b^m_\delta\in B^{\min(c_\delta\setminus (\alpha^\delta_{m-1}+1))}
\setminus B^{\alpha^\delta_{m-1}}$, hence
$\alpha^\delta_{m-1}\in {\rm Inv}_{\bar{B}}(b_\delta^0)$ by virtue of $m$.
Now use Note \ref{observe}(1).
$\eop_{\ref{exact}}$ $\eop_{\ref{main}}$
\end{Proof of the Subclaim}
\end{Proof of the Main Claim}

\begin{Claim}\label{preservation} Suppose that $\bar{A}=\langle A_\delta:\,\delta
\in S\rangle$ is as in the statement of the Main Claim, and $B=B[\bar{A}]$ and
$\bar{B}$ are obtained as in the Main Claim. Further
suppose that $f:\,B\into B^\ast$ is a
c-embedding,while
$\bar{B}^\ast=\langle B^\alpha_\ast:\,\alpha<\lambda\rangle$
is a filtration of $B^\ast$ (so $\card{B^\ast}=\lambda$).

\underline{Then} there is a club $E$ of $\lambda$ such that for every
$\delta\in S$ with the property $c_\delta\subseteq E$
we have
\[
A_\delta={\rm Inv}_{\bar{B}}(b^0_\delta,c_\delta)={\rm Inv}_{\bar{B}^\ast}(
f(b^0_\delta),c_\delta).
\]
\end{Claim}

\begin{Proof of the Claim} Let
\[
E\deq\{\alpha<\lambda:\,
B_\ast^\alpha\cap f``B=f``B^\alpha\}.
\]
Hence $E$ is a club of $\lambda$.

Suppose now that $\delta\in S$ and $c_\delta\subseteq E$. Clearly
\[
f(b^m_\delta)\in B^{\min(c_\delta\setminus (\alpha^\delta_{m-1}+1))}_\ast
\setminus B^{\alpha^\delta_{m-1}}_\ast,
\]
by the definition of $E$,
and the rest follows as in the proof of $A_\delta=
{\rm Inv}_{\bar{B}}(b^0_\delta,c_\delta)$.
$\eop_{\ref{preservation}}$
\end{Proof of the Claim}

{\em Proof of the Theorem finished.}
Suppose that $B^\ast$ is a c-algebra of size $\lambda$
and $\bar{B}^\ast$ any filtration of $B^\ast$. As $\lambda<2^{\aleph_0}$,
we can choose for each $\delta\in S$ an unbounded $A_\delta\subseteq c_\delta$
of order type $\omega$,
such that for
no $b\in B^\ast\setminus B_\ast^\delta$
do we have
${\rm Inv}_{\bar{B}^\ast}
(b,c_\delta)=A_\delta$.
Let $\bar{A}\deq\langle A_\delta:\,\delta\in S\rangle$ and $B\deq B[\bar{A}]$.

Suppose $f:\,B\into B^\ast$ is a c-embedding. Let $E$ be
a club of $\lambda$ guaranteed by Claim \ref{preservation}, and
let $\delta\in S$ be such that $c_\delta\subseteq E$. Hence
${\rm Inv}_{\bar{B}^\ast}
(f(b^0_\delta),c_\delta)=A_\delta$, a contradiction.

The part of the theorem involving $\mu<2^{\aleph_0}$ is proved similarly.
$\eop_{\ref{theorem}}$
\end{Proof}

\begin{Theorem}\label{mali} (1) Suppose that $\clubsuit_{\rm club}+\neg CH$
holds.

\underline{Then} there is no c-universal c-algebra of size $\aleph_1$.
Moreover for no $\mu<2^{\aleph_0}$ are there $\mu$ c-algebras
of size $\aleph_1$ such that every c-algebra of size
$\le\aleph_1$ c-embeds into one of them.

{\noindent (2)} Suppose that either $\lambda=\cf(\lambda)>\aleph_1$
satisfies $2^{\aleph_0}>\lambda$, or $\lambda=\aleph_1$ and
$\clubsuit_{\rm club}+\neg CH$
holds.

\underline{Then} there is no almost c-algebra of size $\lambda$
which is c-universal for almost c-algebras of size $\le \lambda$, or even for
c-algebras of size $\le \lambda$.
Moreover no $\mu<2^{\aleph_0}$ almost c-algebras
of size $\lambda$ taken together have that property.
$\eop_{\ref{mali}}$
\end{Theorem}

\begin{Proof} The same proof as that of Theorem \ref{theorem}.
\end{Proof}
%\input axioms.tex
%\input conventions.tex
\eject

\begin{thebibliography}{xxxxxxx}

\bibitem[Be]{Bell} M. Bell, {\em Universal Uniform Eberlein Compact Spaces},
submitted.

\bibitem[BeRuWa]{Rudin} Y. Benyamini, M.E. Rudin and M. Wage, {\em Continuous
Images of Weakly Compact Subsets of Banach Spaces}, Pacific Journal of
Mathematics, vol. 70, No. 2, 1977, pg. 309-324.

\bibitem[KjSh 409]{KjSh 409} M. Kojman and S. Shelah, {\em Nonexistence of universal
orders in many cardinals}, Journal of Symbolic Logic 57 (1992), 875-891.

\bibitem[KjSh 455]{KjSh 455} M. Kojman and S. Shelah, {\em Universal Abelian
Groups}, Israel Journal of Mathematics, 1995.

\bibitem[Sh g]{Sh g} S. Shelah, {\em Cardinal Arithmetic}, Oxford University
Press 1994.

\bibitem[Sh e]{Sh e} S. Shelah, {\em Universal classes}, in preparation.

\bibitem[Sh 175]{Sh 175} S. Shelah, {\em Universal graphs without instances of
$CH$}, Annals of Pure and Applied Logic vol. 26 (1984), pg. 75-87.

\bibitem[Sh 175A]{Sh 175A} S. Shelah, {\em Universal graphs without instances of
$CH$: Revisited}, Israel Journal of Math, vol. 70, No.1 (1990), pg. 69-81.

\end{thebibliography}
%\eject
%\input private.tex
\end{document}





