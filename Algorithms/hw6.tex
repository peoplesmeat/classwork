\documentclass[11pt,fleqn]{article}
\usepackage{latexsym,epsf,epsfig}
\usepackage{amsmath,amsthm}
\usepackage{xy}
\input xy
\xyoption{all}
\begin{document}
\newcommand{\mbf}[1]{\mbox{{\bfseries #1}}}
\newcommand{\N}{\mbf{N}}
\renewcommand{\O}{\mbf{O}}

\noindent Bill Davis \\
605.421 

\begin{enumerate}
\item 30.1-1 \\
\[
A \times B = c_{0} + c_{1}x + c_{2}x^{2} + c_{3}x^{3} + c_{4}x^{4} + c_{5}x^{5} +c_{6}x^{6} 
\]

\[c_{0} = a_{0}b_{0} = (3)(-10) = (-30) \]
\[c_{1} = a_{0}b_{1}+a_{1}b_{0}=1(3)+(-10)(-6) = 63\]
\[c_{2} = a_{0}b_{2}+a_{1}b_{1}+a_{2}b_{0} = (3)(-1) + -6(1) + 0(-10) = -9\]
\[...\]
\[=56x^{6} -8x^{5} - 42x^{4} - 53x^{3} - 9x^{2} - 63x - 30\]


 

\item 30.2-2 \\
\[ y_{k} = (6, -2-2i, -2, -2+2i)\]

\item 32-1.4\\
We could do this by searching for each set between gaps. For example imagine we are searching for ab*cd*ef. We search first for ab. If this search is successful we search in the remaining string for cd. Again if this search is successful we search in the remaining string for ef. If each search is successful then we have found the string. 

\item  32.2-1 \\
Generates 2 spurious hits. Both 15 and 92 = 4 mod 11, the same as 26. 

\item 32.3-5 \\
We can approach this in the same way by generating automata that match substrings on the search text delimited by gap characters. Then join these states together with a node that recongnizes anything. Rather then automatically fail in the subsections though, a failed match would simply revert the state to the previous gap character if one exists. 

\end{enumerate}


\end{document}
