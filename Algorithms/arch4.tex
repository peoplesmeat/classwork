\documentclass[11pt,fleqn]{article}
\usepackage{latexsym,epsf,epsfig}
\usepackage{amsmath,amsthm}
\usepackage{xy}
\input xy
\xyoption{all}
\begin{document}
\newcommand{\mbf}[1]{\mbox{{\bfseries #1}}}
\newcommand{\N}{\mbf{N}}
\renewcommand{\O}{\mbf{O}}

\noindent Bill Davis \\
\noindent 605.411 Problem Set 5

\begin{enumerate}
\item %Problem 1
  \begin{enumerate}
  \item Yes, this eliminates 1 bubble, since an instruction can, in the same clock cycle read a register in stage 2 that is being written too by another instruction in stage 5.
  \item The branch history table (BHT) is consulted during the instruction fetch stage.
  \item The decode history table (DHT) is consulted during the instruction decode, and only when the instruction is a branch instruction. 
  \item The BHT is consulted for every instruction. 
  \item The DHT is only consulted for branch instructions.
  
  \end{enumerate}
\item %Problem 2
  \begin{enumerate}
  \item This takes 20 clock cycles to complete
  \item See the attached spreadsheet 
  \item Since this would eliminate the bubbles it was take 12 clock cycles to complete
  
  \end{enumerate}

\item %Problem 3
lui \$4, 0x00EB \\
addi \$5, \$0, 8 \\
NOP \\
ori \$4, \$4, 64 \\
NOP \\
NOP \\
sw \$5, 0(\$4) \\	
sll \$5,\$5, 3 \\
lw \$6, 0(\$4) \\
NOP \\ 
NOP \\
slt \$7, \$6, \$5 \\
NOP \\
sw \$7 , 4(\$5) \\



\item %Problem 4
	\begin{enumerate}
	\item Without data forwarding there would be 2 bubbles
	\item With data forwarding there are 0 bubbles
	
	\end{enumerate}

\item %Problem 5
There are no data dependencies in this sequence. Even though the first and second instruction both use register \$5,  sw merely reads the value and does not alter it. 
\item %Problem 6
\begin{enumerate}
\item 24 Clock Cycles
\item 11 Clock Cycles 
\item 34 Clock Cycles (Since this operates like a pipeline with 15 stages)
\item 21 Clock Cycles 

\end{enumerate}
\end{enumerate}



\end{document}
