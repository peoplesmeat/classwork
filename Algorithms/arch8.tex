\documentclass[11pt,fleqn]{article}
\usepackage{latexsym,epsf,epsfig}
\usepackage{amsmath,amsthm}
\usepackage{xy}
\input xy
\xyoption{all}
\begin{document}
\newcommand{\mbf}[1]{\mbox{{\bfseries #1}}}
\newcommand{\N}{\mbf{N}}
\renewcommand{\O}{\mbf{O}}

\noindent Bill Davis \\
\noindent 605.411 Problem Set 8
\begin{enumerate}
\item %Problem 1
\begin{tabular}{|r|r|r|r|r|}
  \hline
Instruction & Max Cache  & Max Cache  & Max Page Faults & Max Page Faults\\
 & With Align & No Align & With Align &  No Align \\ 
\hline
lw \$t8,4,(\$t0) & 32 & 64 & 1  & 2 \\
lb \$t8,4,(\$t0) & 32 & 32 & 1  & 1 \\
\hline
\end{tabular} 
%.\\*[90pt]

\item %Problem 2
 \begin{enumerate}
 \item See attached diagram.
 \item Since there are 64,000 bits per second, with an interupt happening every 10 bits, there are 6,400 interrupts per second. $\frac{2 Ghz}{6400} = 312500$ clock ticks between interrupts. Since the CPI is 4, this means that 78125 instructions could execute between interrupts. 
 
 \end{enumerate}
\item %Problem 3
    Since this machine needs 16 bytes every clock cycle to avoid a stall, it requires a 32,000 million bytes per second bus. 
\item %Problem 4
\begin{enumerate}
\item There are $ \frac{2^{32}}{16384} = 262144$ page table entry each containing 8 bits plus a 30 bit frame id, rounding up to a 40 bit page table entry for a total of $262144 \times 40 = 10485760$ bytes in the page table.
\item There are $ \frac{1.5 \times 2^{29}}{16384} = 49152$ entries in the inverted page table. Each entry contains 8 bits plus a 32 bit page id, rounding up to a 40 bit page table entry for a total of 1966080 bytes in the inverted page table. 
\end{enumerate}
\item %Problem 5
\begin{enumerate}
\item
\begin{tabular}{|r|r|r|}
  \hline
	 & RAID 3 & RAID 5 \\ 
\hline
  I & Possible & Possible \\
 II & Possible & Possible \\
 III & Possible & Possible \\ 
IV & Not Possible & Not Possible \\
V & Not Possible & Possible \\
\hline
\end{tabular}
\item 
You could successfully update stipe 23 as follows.  \\
1. Read strips 20,21,23, P5. This can all happen at the same time. \\
2. XOR the strips together to recreate strip 22.\\
3. XOR that result with strips 20, 21, and the updated 23 to create an updated P5\\
4. Write P5 back to the disk. \\
5. Write updated strip 23. These two writes can happen at the same time. \\
\end{enumerate}
\end{enumerate}

\end{document}
