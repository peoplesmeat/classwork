\documentclass{article}[12pt]
\usepackage{latexsym}
\begin{document}
\large
\begin{center} 
600.363/463 Solution to Homework 9\\
Posted: 11/14/03\\
Due: Wednesday, 11/20/03\\
\end{center}
\subsection*{26.3-3}
By definition, an augmenting path is simple path $s\leadsto t$ in the residual graph $G_f'$. Since $G$ has no edges between vertices in $L$ and no edges between vertices in $R$, neither does the flow network $G'$ and hence neith does $G'_f$. Also, the only edges involving $s$ or $t$ connect $s$ to $L$ and $R$ to $t$. Note that although edges in $G'$ can go only from $L$ to $R$, edges in $G'_f$ can also go from $R$ to $L$. \\
Thus any augmenting path must go\\
$s\rightarrow L\rightarrow R\rightarrow ... \rightarrow L \rightarrow R\rightarrow t$\\
crossing back and forth between $L$ and $R$ at most as many times as it can do so without using a vertex twice. It contains $s$, $t$, and equal numbers of distinct vertices from $L$ and $R$ --- at most 2+2min($|L|, |R|$) vertices in all. The length of an augmenting path is thus bounded above by 2min($|L|, |R|$)+1.

\subsection*{26-5}
\begin{enumerate}
	\item The capacity of a cut is defined to be the sum of the capacities of the edges crossing it. Since the number of such edge is at most $|E|$, and the capacity of each edge is at most $C$, the capacity of \emph{any} cut of $G$ is at most $C|E|$. 
	\item The capacity of an augmenting path is the minimum capacity of any edge on the path, so we are looking for an augmenting path whose edges \emph{all} have capacity at least $K$. Do a BFS or DFS as usual to find the path, considering only edges with residual capacity at least $K$ (pretend the lower-capacity edges don't exist). This takes $O(V+E)=O(E)$ time. (Note that $|V|=O(E)$ in a flow network.)
	\item Max-Flow-By-Scaling uses the Ford-Fulkerson method. It repeatedly augments the flow along an augmenting path until there are no augmenting paths of capacity greater $\geq$1. Since all the capacities are integers, and the capacity of an augmenting path is positive, this means that there are no augmenting paths whatsoever in the residual graph. Thus, by the max-flow min-cut theorem, Max-Flow-By-Scaling returns a maximum flow. 
	\item The first time line 4 is executed, the capacity of any edge in $G_f$ equals its capacity in $G$, and by part (1) the capacity of a minimum cut of $G$ is at most $C|E|$. Intially $K=2^{\lfloor\lg C\rfloor}$, hence $2K = 2\cdot 2^{\lfloor\lg C\rfloor}=2^{\lfloor\lg C\rfloor+1}>2^{\lg C}=C$. So the capacity of a minimum cut of $G_f$ is initially less than $2K|E|$. \\
	
\begin{itemize}
	\item The other times line 4 is executed, $K$ has just been halved, so the capacity of a cut of $G_f$ is at lost $2K|E|$ at line 4 iff that capacity was at most $K|E|$ when the \textbf{while} loop of lines 5-6 last terminated. So we want to sow that when line 7 is reached, the capacity of a minimum cut of $G_f$ is most $K|E|$. \\
	Let $G_f$ be the residual network when line 7 is reached. \\
	$\neg \exists$ augmenting path of capacity $\geq$ K in $G_f$\\
	$\Leftarrow$ max flow $f'$ in $G_f$ has value $|f'|<K|E|$\\
	$\Leftarrow$ min cut in $G_f$ has capacity $<K|E|$
\end{itemize}

	\item By part (4), when line 4 is reached the capacity of a minimum cut of $G_f$ is at most $2K|E|$, and the maximum flow in $G_f$ is at most $2K|E|$. By an extension of Lemma 26.2, the value of the maximum flow in $G$ equals the value of the current flow in $G$ plus the value of the maximum flow in $G_f$. Therefore the maximum flow in $G$ is at most $2K|E|$ more than the current flow in $G$. Every time the inner \textbf{while} loop finds an augmenting path of capacity at least $K$, the flow in $G$ increases by $\geq K$. Since the flow cannot increase by more than $2K|E|$, the loop executes at most $(2K|E|/K=2|E|$ times. 
	\item The time complexity is dominated by the loop of lines 4-7. (The lines outside the loop take $O(E)$ time.) The outer \textbf{while} loop executes $O(\lg C)$ times, since $K$ is initially $O(C)$ and is halved on each iteration, until $K<1$. By part (5), the inner \textbf{while} loop executes $O(E)$ times for each value of $K$; and by part (2), each iteration takes $O(E)$ time. So the total time is $O(E^2\lg C)$. 
  
	
\end{enumerate}

\end{document}