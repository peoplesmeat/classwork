\documentclass[11pt,fleqn]{article}
\usepackage{latexsym,epsf,epsfig}
\usepackage{amsmath,amsthm, amssymb}
\usepackage{xy}
\input xy
\xyoption{all}
\begin{document}
\newcommand{\mbf}[1]{\mbox{{\bfseries #1}}}
\newcommand{\N}{\mbf{N}}
\renewcommand{\O}{\mbf{O}}

\noindent Bill Davis \\
605.421 

\begin{enumerate}
\item 
6.5-8 Give an $\O(n\lg(k))$-time algorithm to merge $k$ sorted lists with $n$ total elements \\
\\
This can be done by creating a min-heap containing $k$ elements, one from 
each list. We can then merge the lists by taking one min element from the 
heap, replacing it with another from the one of the $k$ lists and repeating. \\
\\
This requires $\O(2n\lg(k))$ since for each element we need to first add it
to the min-heap requiring $\O(\lg(k))$-time, then we need to perform 
HEAP-EXTRACT-MAX, again requiring $\O(\lg(k))$-time. \\
\\
We also need to add another member to the heap nodes indicating which 
list the element originated from, in order to replace it with an element 
from the same list. \\

\item
6-2 Analysis of D-ary heaps \\
\begin{enumerate}
\item 
How to represent a d-ary heap as an array. \\
\\
The root of the tree is A[1], its children are in A[2] through A[d+1]. Given the index $i$ of a node we can compute its PARENT, and $k^{th}$ child by by \\
PARENT(i) = $\lfloor \frac{i-2}{d+1}\rfloor$ \\
To find the $k$-th child of a node i = $ di-d + k + 1 $     \\

\item
Since each level $n$ has $n^{d}$ nodes for a d-ary tree, the height of the tree equals $\lg_{d}(n)$.

\item
EXTRACT-MAX does not need to be altered for d-ary heaps. The only operation that needs to change is MAX-HEAPIFY. Therefor the runtime of EXTRACT-MAX = runtime of MAX-HEAPIFY. This equals $\O(\lg_{d}(n)$ for a d-ary tree of size n

\item
The HEAP-INSERT does not need to be altered for d-ary heaps. Since only parent nodes are examined in HEAP-INCREASE-KEY, each $\lg_{d}(n)$ parent could be compared to the new node resulting in a runtime of  $\O(\lg_{d}(n))$. 

\item
HEAP-INCREASE-KEY can be altered as follows \\
\begin{tabbing}
	If \= key < A[i]  \\
	\> then \= error \\
	A[i] = key \\
	while $i>1$ and A[Parent[i] $<$ A [i]\\
	\> do exchange A[i] and A[Parent[i]\\
	\> \>i = Parent[i]\\
\end{tabbing}
We may need to traverse the whole height of the tree for each call to this function, which as we mentioned before can be done in $\lg_{d}(n)$ time. 


\end{enumerate}

\item 
24.3-2 An example where Dijkstras algorithm fails to produce correct results \\
\xymatrix{
A \ar[dr]^9 \\
& B \ar[r]^5 
& C \ar[ld]^5 \\
& D \ar[r] \ar[u]^{-15} 
& E 
} \\
The proof for Dijkstras fails because equation 24.2 does not hold. In particular
\[
d[y] = \delta(s,y) 
\nRightarrow d[y] 
\leq \delta(s,u) 
\]
If edge lengths can be negative then $\delta(s,u)$ may be less then $\delta(s,y)$.


\end{enumerate}
\end{document}
