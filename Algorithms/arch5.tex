\documentclass[11pt,fleqn]{article}
\usepackage{latexsym,epsf,epsfig}
\usepackage{amsmath,amsthm}
\usepackage{xy}
\input xy
\xyoption{all}
\begin{document}
\newcommand{\mbf}[1]{\mbox{{\bfseries #1}}}
\newcommand{\N}{\mbf{N}}
\renewcommand{\O}{\mbf{O}}

\noindent Bill Davis \\
\noindent 605.411 Problem Set 6

\begin{enumerate}
\item %Problem 1
lwc1 \$f2, x  \\
lwc1 \$f4, y  \\
c.lt.s \$f2, \$f4 //Set condition if f2 $<$ f4 \\
movt.s \$f3, \$f4 //If condition is set mov f4 into f3\\ 
movf.s \$f3, \$f2 //If condition is not set mov f2 into f3\\
swc1 \$f3, z //f3 now contains either x or y, mov that into z\\


\item %Problem 2
This suffers from resource conflicts, since both instructions use the same register operands, and use the ALU. 
\item %Problem 3
  \begin{enumerate}
  \item There are 4! = 24 possible orders. 
  \item It would require 2 bits to identify which of the 4 ways is the least recently used.
  \item This could be done with five bits
  \item A Pseudo-LRU could be done with only three bits. 

\end{enumerate}
\item %Problem 4
\begin{enumerate}
\item 
Fully Associative \\
$|$25 Bit Tag $|$ 7 Bit Offset $|$ \\
The tag is used to see if any line in cache has already stored that line in memory. The offset identifies the byte we are interested in. \\
\\
Direct Mapped\\
$|$20 Bit Tag$|$ 5 Bit Line $|$ 7 Bit Offset $|$ \\
Here we have a match if the line number identified by the middle 5 bits matches the tags on one of the lines in the cache. If it does we can use the 7 bit offset to get the byte we are interested in. \\
\\
Set Associative \\
$|$ 22 Bit Tag $|$ 3 Bit Set $|$ 7 Bit Offset $|$ \\
We try to match on of the 4 lines in the set identified by the middle 3 bits 
to see if any of them match the 22 bit tag. If they do we use the 7 bit offset to obtain the byte of interest. 
\item 
Fully Associative \\
2 Matches (293824 and 293764) (3072 and 3184) \\
\\
Direct Mapped  \\
1 Match (3072 and 2184)
\\
Set Associative \\
2 Matches (293824 and 293764) (3072 and 3184)

\item
Fully Associative - No line replacements neccessary \\
Direct Mapped - 2 Line replacements 2948 replaces 293824, then 293764 replaces 2948 \\
Set Associative - 1 Line replacement, 4088 replaces 2948. \\

\item 
Fully Associative
25 bit Tag for 32 lines = 800 bits \\
1 bit modify per line = 32 bits \\
832 Bits of overhead \\
\\
Direct Mapping
20 bit tag for 32 lins = 640 bits \\
1 bit modify per line = 32 bits \\
672 bits \\
\\
Set Asssociative \\
3 bit psuedo lru for 8 sets = 24 bits \\
22 bit tag for 32 lines = 704 \\
1 bit modify per line = 32 bits \\
760 bits
\end{enumerate}
\end{enumerate}

\end{document}
