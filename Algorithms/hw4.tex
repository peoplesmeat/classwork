\documentclass[11pt,fleqn]{article}
\usepackage{latexsym,epsf,epsfig}
\usepackage{amsmath,amsthm}
\usepackage{xy}
\input xy
\xyoption{all}
\begin{document}
\newcommand{\mbf}[1]{\mbox{{\bfseries #1}}}
\newcommand{\N}{\mbf{N}}
\renewcommand{\O}{\mbf{O}}

\noindent Bill Davis \\
605.421 

\begin{enumerate}
\item 19.1-3 Binary Representation of Binomial Heaps \\
There are $n\choose k$ binary $k-$strings containing exactly $j$ 1's. And since there are also $n\choose k$ nodes per level of a binomial tree, there must be k ones per node per level of a tree when counting the lowest node as 0. \\
\\
Since binomial trees have defined sizes, ie they either have 1,4,8,16 ... nodes,  $2^{0}, 2^{1}, 2^{2}, 2^{3} ...$ , a node of degree k will have subtrees of size $2^{0} + 2^{1} +  2^{2} + ... 2^{k-1} = 2^{k-1}$, therefore it will have k 1's in it's binary representation. 
\item 19.2-5 Binomial-Heap-Minimum \\
Binomial-Heap-Minimum may not work as coded. If all of the nodes in the heap have value $\infty$, then Binomial-Heap-Minimum will return NULL, even though there are elements in the Heap. \\
\\
It can be altered to account for this by, 
\begin{tabbing}
	y $\leftarrow$ \emph{head}[H] // This sets a default return value \\
	x $\leftarrow$ \emph{head}[H]\\ 
	min $\leftarrow \infty$ \\
	whil\=e x $\neq$ NILL \\
	\> do if \= \emph{key}[x] $<$ \emph{min}\\
	\> \> \textbf{do if }\=  \emph{key}[x]$<$ \emph{min} \\
	\> \> \>\textbf{then} \= \emph{min} $\leftarrow$ \emph{key}[x] \\
   \> \> \> \>$y \leftarrow x$ \\
 	\> \> $x \leftarrow$ \emph{sibling}[x] \\
	\textbf{return} y \\
\end{tabbing}
This way if there is an element $< \infty$ we will select it, otherwise we'll return the first value. 

\item 19.2-6 Binomial-Heap-Delete \\
We can alter Binomial-Heap-Delete to work in the case where there is no representation for $-\infty$. \\

\begin{tabbing}

Bino\=mial-Heap-Delete(H,x) \\
	\>y = Binomial-Heap-Extract-Min(H) \\
	\>Binomial-Heap-Decrease-Key(H,x, y-1) \\
	\>Binomail-Heap-Extract-Min(H)\\
	\>Binomial-Insert(y) \\
\end{tabbing}
In this manner they key x always has the lowest value. It simply requires two deletes and an insert, all of which can be done in $\O(\lg{n})$

\item
Minimum spanning tree with binomal heaps\\
A binomial heap can be used to manage both the edge and vertex list. For the edge list the operations are straightforward. Extracting the minimum-weight edge is simply Binomial-Heap-Extract-Min, and $E_{i} \leftarrow E_{i} \bigcup E_{j}$ is the Union operation. \\
\\
Vertex operations are slightly more complicated. Once we extract an edge from the edge list we need to determine which heaps the endpoints belong in. This requires a new operation to search the heap looking for elements. For example, once we extract the minimum-weight edge $(u,v)$ from $E_{i}$, we have to determine which $V_{i}$ and $V_{j}$, $u$ and $v$ belong in. After this we can reuse the union operation to merge  $V_{i}$ and $V_{j}$ if neccessary. \\
\\
The runtime of this algorithm should be E lg(E), since we require approx. E Extract-Min operations which happen in $E$ time. 

\end{enumerate}

\end{document}
