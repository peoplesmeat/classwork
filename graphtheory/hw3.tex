\documentclass{article}
\usepackage{tikz}
\usepackage{skak} 
\usetikzlibrary{arrows,automata}
\usetikzlibrary{positioning}

\pagestyle{empty}

\begin{document}
Bill Davis\\
Homework 3

\begin{enumerate}
  
  \item[3]  c is planar. 
   f is not, a $k_{3,3}$ can be formed from adh, cfi. 
   h are not.  
  
  \item[15]
  \begin{enumerate}
      	\item
      		Euler's formula for a planar graph with connected subgraphs =
      		e-v+1+Total number of unconnected components. 
      	
      		
  \end{enumerate}
  \item[18]
  \begin{enumerate}
      \item 
      Assume we have a connected planar graph with the degree of every vertex
      at least 6. Then $\sum$degrees $\geq 6v$, in which case since
      $\sum$degrees = 2e, then $2e \geq 6v $, in which case $e \geq 3v $, but
      since the graph is planar $e \leq 3v-6$. Since e cannot be both greater
      then 3v and less then 3v-6 there is a contradiction. 
      \item
      By 15b, since eulers formula extends to unconnected planar graphs, (a)
      immediately generalizes to unconnected planar graphs. 
  \end{enumerate}
  \item[20]
  Since all degree of a region is at most k, the sum of the degrees of the all
  regions must be less then kr. But since the sum of the degrees of all the
  regions must equal 2e, then $2e<kr$, and by Eulers formula $2e<k(e-v+2)$
  Then $0<e(k-2)-vk+2k$ and $k(v-2) < e(k-2) $, then $e>\frac{k(v-2)}{(k-2)}$
  \item[2]
  Fort he base case, $1^2 = \frac{(1)(2)(3)}{6}$. For the inductive step
  \[
  \sum_{1}^{n+1} n^2 = (\sum_{1}^{n} n^2) + (n+1)^2
  \]
  By the inductive hypothesis 
  \[
  (\sum_{1}^{n} n^2) + (n+1)^2 = \frac{n(n+1)(2n+1)}{6} + (n+1)^2 
  \]
  By factoring $(n+1)$
  \[
  \frac{n(n+1)(2n+1)}{6} + (n+1)^2 = \frac{(n+1)(2n^2+n+6n+6)}{6} =
  \frac{(n+1)(2n^2+7n+6)}{6}
  \]
  which, when factored 
  \[
  \frac{(n+1)(2n^2+7n+6)}{6} = \frac{(n+1)(n+2)(2n+3)}{6}
  \]
  Which proves the inductive hypothesis. 
  
  \item[16]
  For the base case we can see that $2^5=32 > 25 = 5^2$. For the inductive step
  assume that $2^n>n^2$ for some $n>5$, then $2^{n+1} = 2(2^n)$, where $(n+1)^2
  = n^2+2n+1$. Therefore if $2n+1>2^n$ then $2^{n+1}>(n+1)^2$. And since it is
  the case that $2n+1>2^n$ for all $n\geq 5$ the inductive step is proven. 
  \item[\#A]
  For the base case we can see that $0^3-4(0)+6 = 2(3)$. For the inductive step 
  \begin{eqnarray}
  (n+1)^3-4(n+1)+6 &=& n^3+3n^2+3n+1 - 4n - 4 + 6 \\
  &=& (n^3-4n+6) + 3n^2+3n-3 \\
  &=& 3k + 3(n^2+n-1) \\
  &=& 3(k+n^2+n-1)  
  \end{eqnarray}
  Where 3 uses the inductive hypothesis that $(n^3-4n+6) = 3k$ for some k.
  Therefore the inductive step is proven. 
  \item[4]
  The total number of books that must be chosen is $10+8+11+12=41$
  \item[5]
  This can be modeled by a graph of 20 vertices with 48 edges. If every person
  knew at least 5 people, then the sum of the degrees would be 100, and since
  this does not equal 98, at least one person must know 4 or fewer people. 
  \item[10]
  If we add pairs of days (1,2),(3,4),(5,6),(7,8),(9,10),(11,12), there are 6
  such pairs. If each pair has a sum less then 17, then the total is 6(17) = 96.
  Since the computer was used for 99 hours, at least one pair must have a total
  greater then or equal to 17. 
  \item[18]
  After fixing the bottom disk, there are 20 possible rotations for the disks to
  match. We'll try to count the total number of matches over all rotations. 
  Each 1 or 0 on the top disk will match a total of 10 times over the 20
  rotations,  therefore the total number of matches is 200. If we assume that 
  each rotation has 9 or fewer matches then there will only be a maximum number
  of 180 total matches, which is a contradiction. Therefore at least one
  rotation  must match at least 10 items.
  \item[20]
  I didn't come up with this solution, but it seems very simple once you see
  it. For any $k_6$, at least 3 edges at any vertex must be of the same color,
  say red. Consider a vertex v and 3 edges connected to vertices, a,b,c. If one
  of the edges connecting these is red then there is a red triangle, otherwise
  there must be blue edges connecting a,b,c. Either way there is a triangle. 
\end{enumerate}

\end{document}