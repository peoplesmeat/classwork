\documentclass[11pt,fleqn]{article}
\usepackage{latexsym,epsf,epsfig}
\usepackage{amsmath,amsthm}
\usepackage{xy}
\input xy
\xyoption{all}
\begin{document}
\noindent Bill Davis \\
\noindent 605.441 Problem Set 8
\begin{enumerate}
\item 17.9 %Problem 1 
A serial schedule is one in which operations are completed in consecutive order, without and task switching. That is, no two operations are taking place in an interleaved fasion. A serializable schedule is a schedule, that perhaps interleaves operations, that can be transformed into a serial schedule through a series of conflict equivalent transactions. \\
A serial schedule is considered correct because any set of operations are designed to move the database from one correct state to another. It follows that these operations are correct. A serializable schedule is correct because it is equivalent to a serial schedule. 

\item  17.22 
c. This is conflict serializable. The equivalent serial schedule is $r_{2}(X), r_{3}(X), w_{3}(X) ,r_{1}(X), w_{1}(X)$ \\
d. This schedule is not serializable. 

\item 18.1 A two-phase locking protocol is a method of guaranteeing serializability. A transaction supports this protocol if all lock operations precede any unlock operations. This divides a transaction into two phases, one in which any necessary locks are gathered, and a second where all the locks are released. This ensures serializability because a transaction will not release any resources it has gathered until it has completed all of its operations. 


\end{enumerate}
\end{document}
