\documentclass[11pt,fleqn]{article}
\usepackage{latexsym}
\usepackage{amsmath,amsthm}
\usepackage{xy}
\input xy
\xyoption{all}
\begin{document}
\newcommand{\mbf}[1]{\mbox{{\bfseries #1}}}
\newcommand{\N}{\mbf{N}}
\renewcommand{\O}{\mbf{O}}

\noindent Bill Davis \\
Homework 6 \\
April 6 2011

\begin{enumerate}

\item
A 2-change does not neccessarily define an exact neighborhood for TSP. For example in the problem below the solution noted is a minimum within a two change, but not within four changes. 

\item
What makes binary integer programming more difficult then linear programming, despite the reduction in the possible values of the variables, is the lack of a nice linear transition function between various points in the feasible region. In particular a way to restrict a variable to take on either a 0 or 1 is to add the constraint that $ x(x-1) = 0$. This will of course only be satisifed when x is 0 or 1, however is nonlinear. 

\item 
Degeneracy is a problem when solving a linear programming problem because it breaks the assumption that at any given vertex in the polytop defined by the problem, you can select a better vertex and move in some direction. When a degenerate vertex is present, at a given vertex there is a move that will leave the algorithm at the same vertex that it's trying to leave. This can lead to cyclical moves, and prevent the simplex algorithm from finding a solution to the problem. 

\begin{quote}
In principle, cycling can occur if there is degeneracy.  In practice,  cycling does not arise, 
but no one really knows why not.  Perhaps it  
does occur, but people assume that the simplex algorithm is just
taking too long for some other reason, and they never discover the 
cycling.
\end{quote}

http://ocw.mit.edu/courses/sloan-school-of-management/
15-053-optimization-methods-in-management-science-spring-2007/tutorials/tut5.pdf

\item
No, even if every vertex in an LP is nondegenerate this does not imply that there exists a unique solution. For example consider the LP defined by $x_1<5$, $x_2<5$ with the cost function $max(x_1,x_2)$. In this example there are multiple optimal solutions along the lines $x=5$ and $x_2=5$. 

\item
We'll need some definitions to show this. First a set is convex iff each distinct pair of points in the set the closed segment between those two points is also in the set. And two a convex polytope is the convex hull of a finite set of points, where the finite set contains the set of extreme points of the polytope. An extreme point is a point not lying on any open line segment contained in the set. A face is the intersection of $P$ and any supporting hyperplane, where a hyperplane is supporting if $P$ is entirely contained within one of the two closed spaces defined by the hyperplane. 

So assume we have a k-dimensional face $F$ of a polytope $P$. Then $F$ is the intersection of a k-dimensional hyperplane and $P$. 

Assume for sake of contradiction that $F$ is not convext, then there exists two points in the face that are unable to be connected via a line. However these two points must be able to be connected in $P$ since $P$ is convex. This implies that at least one interior point of the line segment containing the two points is not present in the intersection of the hyperplane and $P$. However this contradictions the definition of a hyperplane, since any two points in a hyperplane are able to be connected by a line. 



\end{document}
