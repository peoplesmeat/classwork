\documentclass[11pt,fleqn]{article}
\usepackage{latexsym}
\usepackage{amsmath,amsthm}
\usepackage{xy}
\input xy
\xyoption{all}
\usepackage{url}
\begin{document}
\newcommand{\mbf}[1]{\mbox{{\bfseries #1}}}
\newcommand{\N}{\mbf{N}}
\renewcommand{\O}{\mbf{O}}

\noindent Bill Davis \\
Homework 7 \\
April 13 2011

There is two side to a MapReduce implementation. First are generic
MapReduce algorithms that run on the system. These can be used to do anything
from compute eigenvectors of sparse matrices to calculting a matrix product to
computing a minimum spanning tree. However, in order for an algorithm to run
efficiently on a parallel machine, there has to be an efficient parallel
algorithm. The class NC captures the concept of efficient parallel algorithms.
Since NC is a subset of P, we can wonder if there are problems that are in P but
not in NC. Of course there is no proof, but the general intuition is that the
class P-Complete captures algorithms which are in P but not in NC. And linear
programming, optimization falls into the class P-Complete. This implies that
there is no easily parallelizable algorithm which can run on a cluster and solve
a linear/nonlinear optimization problem. 

There is second side the MapReduce equation and that is the
efficient running of a given MapReduce algorthim. This can easily be formulated
as an optimization problem, that is given an input set and a MapReduce program
what are the steps to execute the program while minimizing the load on the
system. In particular its neccessary to map every input file/tuple to a
particular machine in order to be Mapped, and then to allocate reducers
across the cluster to process the resulting tuples. The constraints here are
network/cpu capacity. This is similar to a max-flow problem. In essense you're
trying to maximize the throughput of input tuples across the system. 

\begin{thebibliography}{9}

\bibitem{fisher1}
  Michael J. Fischer, Xueyuan Su, and Yitong Yin
  \emph{Assigning Tasks For Efficiency In Hadoop}.
  \url{http://cs.yale.edu/homes/xs45/pdf/fsy-spaa2010-slides.pdf}
\bibitem{wikipedia1}
	Wikipedia, the free encyclopedia
	\emph{P-complete} 
	\url{http://en.wikipedia.org/wiki/P-complete}

\end{thebibliography}
\end{document}